% Awesome Source CV LaTeX Template
%
% This template has been downloaded from:
% https://github.com/darwiin/awesome-neue-latex-cv
%
% Author:
% Christophe Roger
%
% Template license:
% CC BY-SA 4.0 (https://creativecommons.org/licenses/by-sa/4.0/)

%Section: Project
\sectionTitle{Projects}{\faLaptop}

\begin{projects}
	\project
	{LrSensei (Adobe)}{2020 - }
	{Computer Vision Project Ranging From Best Photos, Auto Stacking to Panoptic Segmentation}
	{\\Lightroom Sensei is a multiple-goal Computer Vision Project. Currently, I am working on Best Photos, Visual Similarity, and Panoptic Segmentation. My goal is to enhance and develop models, look for its deployment, try to find out suitable metrics, establish requirements by Users, and look for guarantees that different models provide. This work also involves choosing the deployment choices for the model -like can we put on client-side or on the server, how effectively jobs can be distributed on CPUs and GPUs. }
	{Torch, ONNX, CoreML, Feature Pyramids, Contrastive Learning}
				
	\project
	{\\LrPerformance (Adobe)}{2016 - 2020}
	{Performance Enhancement and User Analysis for Lightroom Classic}
	{\\Lightroom Performance Enhancement had four major directions -(a) Efficient Algorithms, (b) Deep Parallelism, (c) Better Resource management, and (d) Machine Learning to achieve goals associated with (a)-(c). In an adversarial environment where User requirements are unknown, where Computation is costly (as Lightroom is a Photography Application) and Operations are done on multiple assets at the same time(here assets are Images), the challenges become multi-faceted. We successfully used Common Table Expressions of Database and System Parallelism to gain Performance. Thereafter we deployed Classical Reinforcement Learning based solutions, Markov Decision Process and Bandits to estimate resources. This is a very new paradigm, which has been successfully productized. Some of the developments, for example for Grid, involved estimating Geometry shifts using Weibull Distribution. Other than Performance, we used User Log data to predict future features, User churn rate, Probability of user to convert. We even tried to do Garbage Collection in Lua using Reinforcement Learning.}
	{C++, Lua, SqLite, Markov Decision Process, Reinforcement Learning, Keras, LSTM, Hive, Hadoop}
	
	\project
	{\\Devnagiri OCR (Qualcomm)}{2011 - 2015}
	{Optical Character Recognition for Devnagiri}
	{\\. We developed Optical Character Recognizer for Indic (Devnagiri) Characters. We developed both Char Decoders and Word Decoders. This solution was developed for Mobile Devices hence challenges were on performance side as well. One of my goals was to develop cache aware vector routines. I developed Algorithm for Chandrabindu development and also wrote modules for Word Decoders.}
	{C++, Computer Vision, Machine Learning, Android}	

\end{projects}